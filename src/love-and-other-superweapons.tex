\environment env_book

\starttext

\title{\GameName}

\subtitle{\goto{CC-BY 4.0}[url(https://creativecommons.org/licenses/by/4.0/)] by \goto{Astral Frontier}[url(https://astralfrontier.itch.io)]}

The Old Provinces are the lands that survived the fall of ancient Zebulon.
The magitek superweapons that kept the provinces safe from invasion were restricted via DNA lock to the provincial governors and their families.
Of course nobody in today's kingdoms knows anything about DNA.
They just know that maintaining the royal bloodline makes the weapons work.
And the proof of royal blood is the activation ceremony, when the successor pushes the button.
Of course, nobody knows whether it's going to be a harmless test firing or the first strike in a new war until it happens...

The Heir's upcoming 21st birthday marks the date of their activation ceremony.
The kingdom's dignity depends on this going well.
Behind the scenes, though, things like this never do.

You can incorporate this scenario into an existing game system. To play as a standalone game, a brief set of rules have been included at the end.

\section{Player Roles}

All but one player will be {\gameterm "Wild Cards"}: the characters who end up meddling in the situation for their own reasons.
They can be confidantes of the other characters, agents provocateurs for one or more governments,
or merely well-informed and motivated adventurers.
Individual Wild Cards may have their own goals, but should have reasons to cooperate at first.

One player makes decisions about what's really going on behind the scenes, and narrates for any non-Wild Card characters involved in the succession.
This player is known as the {\gameterm "Secret Keeper"}, or "SK".

\page

\section{Actors}

The Secret Keeper will choose characters, called {\gameterm "Actors"}, from the following list to include in the scenario.
The Heir is mandatory; all others are optional.
Character species, heritage, gender, etc. may be assigned to taste.

\startitemize[4,packed]
\item {\gameterm The Heir}: the child of the King and Queen, whose birthday is the catalyst for events.
\item {\gameterm The King and the Queen}: father and mother to the Heir. Responsible for policy decisions in the kingdom, as well as moral and cultural leadership. The King and Queen are separate characters, with their own private agendas.
\item {\gameterm The Chamberlain}: a senior functionary in the kingdom's government. Responsible for much of the day-to-day operations and logistics of rule.
\item {\gameterm The Companion}: someone associated with the Heir, like a best friend, confidante, or even a secret lover.
\item {\gameterm The Rival}: a sibling to the Heir, an unrecognized bastard or cadet branch of the royal line, or anyone else who thinks they're capable of activating the weapon and being recognized - whether that's really true or not.
\item {\gameterm The Pacifist}: a demagogue whose movement calls for the end of the use of such weapons and the normalization of peaceful alliances.
\item {\gameterm The Observer}: a representative of a neighboring kingdom. Sent to watch the succession ceremony and report back. May be royalty themselves, or merely a messenger. Outwardly, their kingdom is either neutral or friendly toward this one, but who knows what they (or their rulers) really think?
\item {\gameterm The Antagonist}: a representative of a hostile neighboring kingdom. Their rulers secretly believe that this kingdom intends to fire the superweapon for real, targeting them in a surprise attack. Like the Observer, may be royalty themselves, or something else.
\item {\gameterm The Soldier}: a member of the armed forces of this kingdom, or another one. They may associate with the Chamberlain, the Observer, or the Antagonist based on country of origin. They realize what will happen if war happens.
\stopitemize

\page

\section{Agendas}

The SK should assign an {\gameterm Agenda} to each Actor.
Discovering these Agendas, and supporting or sabotaging them, is the goal of the Wild Cards.
The SK can supply Agendas that aren't on this list.

The SK should assign one or more emotional motives that fuel the agenda:
pride, fear, jealousy, disappointment, and so on.
But it's important to answer one other question about each pairing:
{\it who or what does this Actor love, that makes them feel this way, and pursue this Agenda?}
This could be country, or family, or someone or something else.
But each Actor loves something.

Actors' Agendas can change during play, as the Wild Cards intervene.

\startitemize[4,packed]
\item Ensure the succession happens at any cost
\item Cancel the succession ceremony (e.g. by thwarting some important part of the ceremony, kidnapping the Heir, or the Heir eloping with another character)
\item Ensure succession happens, but not as originally planned (e.g. introduce another candidate, fire the weapon at a real target)
\item Embarrass, humiliate, or destroy another character
\stopitemize

Example Actor/Agenda pairings:

\startitemize[4,packed]
\item The Heir wants the succession to happen, out of loyalty to their parents
\item The King prefers another candidate and will support them if feasible
\item The Queen wants the succession to happen, out of pride
\item The Chamberlain wants to humiliate or defeat the Antagonist, out of loyalty to the kingdom
\item The Rival wants to embarrass or humiliate the Heir, out of loathing
\item The Pacifist wants the succession to happen, out of vigilance, as the Heir seems receptive to their movement's ideals
\item The Observer wants to ensure the succession happens, but with the Rival as the candidate, out of love
\item The Antagonist wants to cancel the succession ceremony, out of fear
\item The Soldier wants the succession to happen, but firing the weapon for real, out of fear of a rival nation
\item The Companion wants to cancel the succession by convincing the Heir that the Pacifist's position is best
\stopitemize

\page

\section{Playing the Game}

The game begins with the Wild Cards meeting up and comparing notes on what they know so far, then deciding what action to take.
The game ends when the decision whether to fire the superweapon has been made, and how, and the kingdom's succession is determined.

The SK should decide which Actors are included in the game, assign Agendas to each of those Actors, and then enact those Agendas through narration.

To create a Wild Card character, write 3-6 things they are notable for,
like "Breaking and Entering", "Political Maneuvering", "Empathy", or "Maid at the Castle".
Assign scores of +2, +1, -1, or -2 to each trait such that all traits sum to zero.
Higher numbers mean you are better at support, lower numbers mean you are better at sabotage.

When rolling, add up the bonuses for any relevant trait(s).
If multiple Wild Cards cooperate on a task, add all relevant traits.

When a Wild Card {\gameterm gathers intel on another character's plans}, describe how and roll 2d6 +traits.

\startitemize[4,packed]
\item On a 5 or less, the plans are in shambles or headed for disaster.
\item On a 6-8, it's complicated. Multiple characters are involved.
\item On a 9 or higher, the plans look like they're going to succeed unless someone intervenes.
\stopitemize

When a Wild Card {\gameterm intervenes in an Actor's ongoing Agenda}, describe how and roll 2d6 +traits.

\startitemize[4,packed]
\item On a 5 or less, the situation is sabotaged. The other character's efforts are thwarted in some important way.
\item On a 6-8, the situation is complicated. The SK describes some new element that's been added, such as another character suddenly becoming involved.
\item On a 9 or higher, the situation is supported. The other character's efforts will be more successful or fruitful.
\stopitemize

When a Wild Card {\gameterm gains the ongoing support of an Actor}, write that character's name as a trait.
The SK will assign a value (+2, +1, -1, -2) to that trait based on their agenda.
If that character's support is lost, remove the trait.

Wild Cards are assumed to be capable individuals; if a player proposes an action that doesn't fall under one of these categories,
that action should be assumed to succeed.

\stoptext